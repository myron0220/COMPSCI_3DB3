\documentclass[11pt,fleqn]{exam}

\setlength {\topmargin} {-.15in}
\setlength {\textheight} {8.6in}
\usepackage{enumerate}
\usepackage{amsmath}
\usepackage{amssymb}
\usepackage{listings}
\usepackage{graphicx}
\newcommand{\nn}{~\newline \noindent }
\newcommand{\mn}[1]{\mbox{\sf #1}}
\newcommand{\sbs}{\subsection*}

\begin{document}
	
	\begin{center}
		
		{\large \textbf{COMPSCI 3DB3}}\\[2mm]
		{\huge \textbf{Assignment 3 Report}}\\[6mm]
		{\large \textbf{Mingzhe Wang}}\\[2mm]
		{\large \textbf{McMaster University}}\\[6mm]
		{\large \today}
		
	\end{center}
	
\medskip
\section*{Part one}
\subsection*{Subscriber}
subscriber(\underline{username}: VARCHAR(255), \underline{number}: INT, email: VARCHAR(255), hash: BINARY(512), salt: BINARY(512)).
\begin{itemize}
\item The username and number should be the primary key. 
\item \mn{VARCHAR(255)} is selected as the type for email, because an email address consists of variable length character.
\item \mn{BINARY(512)} is the type of hash and salt value, because they are stored in fixed-length binary strings.
\item Note: because db2 seems not to support \mn{BINARY(512)} type, in the code, we use VARCHAR(512) instead.
\end{itemize}

\subsection*{Friend\_Of}
friend\_of(\underline{fname}: VARCHAR(255), \underline{fnumber}: INT, \underline{tname}: VARCHAR(255), \underline{tnumber}: INT).
\begin{itemize}
\item fname, fnumber, tname, tnumber should all be not null, primary key, and reference from subscriber(username) and subscriber(number).
\end{itemize}

\subsection*{Review}
review(\underline{uname}: VARCHAR(255), \underline{unumber}: INT,  \underline{revision}: INT, \underline{[Forfilm's primary key(s)]}: [corresponding type(s)], score: INT, timestamp: TIMESTAMP).
\begin{itemize}
\item If we assume the score is in the range $[0, 10]$, the we should mark this as a constraint and check it.
\item We may need to update the ForFilm part later, as this information haven't been constructed.
\end{itemize}

\subsection*{VideoReview}
video\_review(\underline{uname}: VARCHAR(255), \underline{unumber}: INT,  \underline{revision}: INT, \underline{[Forfilm's primary key(s)]}: [corresponding type(s)], video: BLOB).
\begin{itemize}
\item We use ER method to construct ISA, because a review could have both kinds of reviews (video and text).
\item The uname, unumber, revision, and [Forfilm's primary key(s)] are foreign keys pointing to subscriber's.
\end{itemize}

\subsection*{TextReview}
text\_review(\underline{uname}: VARCHAR(255), \underline{unumber}: INT,  \underline{revision}: INT, \underline{[Forfilm's primary key(s)]}: [corresponding type(s)], description: CLOB).
\begin{itemize}
\item Similar to videoreview.
\end{itemize}

\subsection*{Reaction}
reaction(\underline{id}: INT, byuname: VARCHAR(255),  byunumber: INT, title: VARCHAR(255), content: CLOB).
\begin{itemize}
\item We use ER method here, because the ThreadR entity could have a relation with Reaction.
\item We store the by subscriber's username and number here as foreign keys referencing subscriber(username, number), as reaction only participates exactly once in the``by" relation. It should also has a NOT NULL constraint.
\end{itemize}

\subsection*{ThreadR}
reaction(\underline{id}: INT, onid: INT)
\begin{itemize}
\item We need id for ISA relation and onid for On\_Ration relation. Both id and onid should reference to the reaction(id).
\end{itemize}

\subsection*{ReviewR}
review\_r(\underline{id}: INT, [Forfilm's primary key(s)]: [corresponding type(s)])
\begin{itemize}
\item We store the On\_Review relation as foreign keys referencing to Review's primary keys, because the ReviewR entity participates exactly once in this relation.  It should also has a NOT NULL constraint.
\end{itemize}

\section*{Part two}
\subsection*{person}
person(\underline{id}: INT, name: VARCHAR(255), birthdate\textsubscript{optional}: DATE)
\subsection*{film}
film(\underline{title}: VARCHAR(255), \underline{year}: INT, \underline{creator}: INT, duration: INTERVAL, budget: DECIMAL(50,2))
\begin{itemize}
\item We treat creator as a strict one-to-many relationship between film and person, so we store it in film.
\item foreign key constrain: creator references to person(id).
\end{itemize}
\subsection*{film\_info (view)}
(view) film\_info(\underline{title}: VARCHAR(255), \underline{year}: INT, \underline{creator}: INT, duration: INTERVAL, budget: DECIMAL(50,2))
\begin{itemize}
\item this is a view, so we should use "select" sytax later.
\end{itemize}
\subsection*{role\_as}
role\_as(\underline{pid}: INT, \underline{ftitle}: VARCHAR(255), \underline{fyear}: INT, \underline{fcreator}: INT, role: VARCHAR(20))
\begin{itemize}
\item we assume play\_role\_as as a many-to-many relation between person and film. The relationship should has an attribute role to denote the person's current role in the file. For cases where a person can perform many roles in a film, just add multiple instances with different role attributes.
\end{itemize}
\subsection*{constraint}
this can be implemented using multi-table constraints by:\\
CREATE TABLE film(\\
title VARCHAR(255) ...,\\
year INT ...,\\
creater INT ..., \\
CHECK (creater IN (SELECT r.id FROM roleas r WHERE (r.ftitle = title AND r.fyear = year AND r.fcreator = creator) AND r.role = "director"));

\section*{Note}
By now, we know what are the primary keys in film, so we need to replace every occurrence of ``[Forfilm's primary key(s)]: [corresponding type(s)]'', that is:


\subsection*{film}
\underline{title}: VARCHAR(255), \underline{year}: INT, \underline{creator}: INT, 


\subsubsection*{Review}
review(\underline{uname}: VARCHAR(255), \underline{unumber}: INT,  \underline{revision}: INT, \underline{ftitle}: VARCHAR(255), \underline{fyear}: INT, \underline{fcreator}: INT, score: INT, timestamp: TIMESTAMP).
\subsubsection*{VideoReview}
video\_review(\underline{uname}: VARCHAR(255), \underline{unumber}: INT,  \underline{revision}: INT, \underline{ftitle}: VARCHAR(255), \underline{fyear}: INT, \underline{fcreator}: INT,  video: BLOB).
\subsubsection*{TextReview}
text\_review(\underline{uname}: VARCHAR(255), \underline{unumber}: INT,  \underline{revision}: INT, \underline{ftitle}: VARCHAR(255), \underline{fyear}: INT, \underline{fcreator}: INT,  description: CLOB).
\subsubsection*{ReviewR}
review\_r(\underline{id}: INT,  uname: VARCHAR(255), unumber: INT, revision: INT, ftitle: VARCHAR(255), fyear: INT, fcreator: INT)








		
	
	



\end{document}


