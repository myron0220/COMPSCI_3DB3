\documentclass[11pt,fleqn]{exam}

\setlength {\topmargin} {-.15in}
\setlength {\textheight} {8.6in}
\usepackage{enumerate}
\usepackage{amsmath}
\usepackage{amssymb}
\usepackage{listings}
\usepackage{graphicx}
\newcommand{\nn}{~\newline \noindent }
\newcommand{\mn}[1]{\mbox{\sf #1}}
\newcommand{\sbs}{\subsection*}

\begin{document}
	
	\begin{center}
		
		{\large \textbf{COMPSCI 3DB3}}\\[2mm]
		{\huge \textbf{Assignment 1}}\\[6mm]
		{\large \textbf{Mingzhe Wang}}\\[2mm]
		{\large \textbf{McMaster University}}\\[6mm]
		{\large \today}
		
	\end{center}
	
	\medskip
		
	
	\section*{Analysis}
	This analysis will briefly show how all the requirements are being met and how they will be translated to the final ER diagram.  We will go through this process for all import sentences based on their appearance order in the description. 
	\sbs{Each cinema has a unique name and an address.}
	\mn{Cinema} is an entity. It has two attributes \mn{current name} and \mn{address}, with \mn{address} being its primary key.
	\sbs{In the past, cinema names have changed (e.g., due to renovations and grant reopenings). }
	This reminds us to add a new attribute names \mn{past name} to \mn{Cinema} entity.
	\sbs{Per cinema, the system will also maintain information per room. }
	\mn{Room} should be a weak entity of cinema.
	\sbs{Each room has a unique number within the cinema (e.g., smaller cinemas have Room one to Room five). }
	The weak entity \mn{Room} should have an \mn{room number} attribute as the partial key.
	\sbs{Per room, the system can keep track of the characteristics of the cinema installation.  These characteristics include \ldots}
	\mn{Screen}, \mn{Projector}, \mn{Sound system} should be weak entities of the room, because they all have multiple attributes. We also assume there is an unique \mn{equipment id} for each of these equipment to act as their partial key. The available \mn{accessibility} should be an attribute, as it only has an atom information.
	\sbs{Finally, per room the system also needs to know the exact seat arrangement, as the online system will allow customers to order tickets for specific seats.}
	This reminds us to has \mn{Seat} as an weak entity for \mn{Room}.
	\sbs{Hence, the system keeps track, per room, of each seat and—per seat—its location in the room, the row it is in, and the seat number (within the row).}
	The \mn{Seat} should have two attributes \mn{row} and \mn{number}, with both acting as their partial key for distinguishing.
	\sbs{Furthermore, some seats are standard reserved for disabled people.}
	All seat should have an additional \mn{disable} attribute, which is a boolean type.
	\sbs{Second, the system needs to store information for each screening.}
	\mn{Screening} is an entity.
	\sbs{Each screening is assigned a single room and timeslot.}
	This rules the relationship between \mn{Room} and \mn{Screening} as many (partial) - to - one (total). We suggest attach the \mn{start time} and \mn{end time} as an attribute to \mn{Screening}, as it added to the relationship, it could easily mix things up. In addition, here we suggest adding an unique \mn{screening number} (primary key) for each screening with different time slot, this can be easily implemented by automatically generated by system.
	\sbs{the system distinguishes between three types of screenings: normal screenings will show a single film and will be offered via normal ticket prices; special screenings can show one-or-more films with a special ticket price, e.g., a marathon with a higher ticket price or a classic film at a reduced ticket price; and private screenings (as part of hired rooms).}
	We have three sub entity by ISA relation to \mn{Screening}: \mn{Normal Screening}, \mn{Special Screening}, and \mn{Private Screening}. Due to the principle of reducing redundancy, we only need one attribute \mn{price} attached to their super entity \mn{Screening}. Different prices can be expressed just by number.
	\sbs{This film information is provided by the film distributors in a standard format: for now, the system represents this external information via an entity Film with an attribute fid.}
	\mn{Film} is an entity with the attribute \mn{fid} as its primary key.
	\sbs{For each public screening, the system keeps track which films are shown. }
	\mn{Film} has relationship \mn{showing} with \mn{Screening}. It should be noticed, for different type of \mn{Screening},  they can show multiple or one films. This kind of relation is showed in ER diagram: each \mn{Normal Screening} can show exactly one film, while each other types of screening should show at least one \mn{Film} (many, total).
	\sbs{Via an (online) sale, customers can buy one or more tickets for a specific screening and that are assigned a seat on sale. }
	\mn{Customer} is an entity, who participates many times in a \mn{Sale} relationship. We assume each customer should have an unique \mn{uid} generated by the system. In practice, we may also need to create attributes like \mn{last name}, \mn{first name}, and \mn{phone number} for them.
	\sbs{For each sale, we keep track of how they are made (online, ticket machine, at the counter) and whether the sale was related to a reservation. }
	\mn{Sale} relationship should have two attributes: \mn{channel} (which means the selling channel e.g. online, ticket machine, or counter) and \mn{is a reservation}.
	\sbs{Customers that do not feel comfortable with paying online can reserve their seats online and buy a ticket for these reservations at the counter (these reservations will be cancelled 45 minutes before the start of the film).}
	This requirement has been met by the \mn{Sale} relationship.
	\sbs{Furthermore, we keep track of the paid price (which can depend on coupons, special actions, payment method, or room and film options).}
	The \mn{Sale} relationship has an unique \mn{transaction number} as the primary key and some attributes such as \mn{paid price}, \mn{coupons}, \mn{special actions}, \mn{payment method}. We omit the room and file options here, because each \mn{Sale} relation already connected with these information in our module. Because \mn{Sale} is an relationship, when attaching an weak entity to it, we should use aggregation here. The details are showed in the ER diagram.
	\sbs{If a screening will show multiple films (as part of special screening), then each of these films will be shown in the same room and the ticket of the customer assign the same seat during each film.}
	First, a new relationship called \mn{Room Assignment} between \mn{Room} and \mn{Screening} is created, which is a many (partial) - to - one (total) relationship. Note: we treat \mn{time slot} as an attribute of \mn{Screening}. That is why the room can participate many times in this relation.
	
	\nn
	Second, the \mn{Seat} entity, instead of \mn{Room} should participate at most once to \mn{Sale} relationship based on the fact that in a particular time slot, a seat can only be owned by one customer. Here, \mn{Sale} should evolve to an tenary relationship among \mn{Seat}, \mn{Screening}, and \mn{Customer}.  However, the using of the tenary relationship can be further optimized to an aggregation between \mn{Customer} and \mn{Screening}, connected with the \mn{Seat} entity.By doing this, we can have a new but import constraint -- The aggregated entity, which we call \mn{Customer\&Screening} can participate exactly once (one, total) in the \mn{Provide} relationship.
	
	\newpage
	\section*{ER Diagram}
	Note: 
	\begin{itemize}
	\item Because in the yEd app, there is no dotted lines to help us distinguish a partial key, we used a blue node for partial key.
	\item Also the thick line is for many, total, as the professor mentioned recently.
	\end{itemize}
	
	\begin{figure}[hbt!]
  	\centering
  	\includegraphics[width=1.0\textwidth]{erdiagram}
	\end{figure}
	
	



\end{document}


